\documentclass[11pt]{article}
\usepackage{amsmath,textcomp,amssymb,geometry,graphicx,tikz,cancel}
\usepackage{algpseudocode,algorithm}
\usepackage[T1]{fontenc}
\pagenumbering{gobble}
\title{BIOE 147 - Detailed Project Description}
\author{Zack Field, Ryan Tsoi}
\date{}

\begin{document}
\maketitle

% \noindent\makebox[\linewidth]{\rule{\paperwidth}{9pt}}
 
 Much of what we have studied in class relies on repressor/activator protein
 and promoter pairs. We have seen that there is a limit to the size of circuit 
 that you can construct, due to the lack of a library of orthogonal proteins.
 While great strides have been made in seeking out new orthogonal repressing/
 activating proteins, it is worth noting that many of the papers that we
 review in lecture still use the same 3 promoter/repressor pairs. 
 It is not a simple task to rationally design our way out of this predicament.
 Since proteins function is dependent upon structure, and the prediction of 
 protein structure is a difficult task (to say the least),
 it would be beneficial to search elsewhere 
 for a class of molecules that can act as a reliable parts family. 

 It is already well known that RNA 
 plays a major role in transcriptional, and translational, regulation within 
 the cell  ~\cite{review}. RNA, unlike protein, has well characterized canonical base pairing 
 interactions.  These interactions dominate the formation of secondary structure. 
 Which makes RNA less recalcitrant to rational design than protein ~\cite{howfolds}. Many tools
 have appeared recently that not only quicly and accurately predict RNA secondary structure,
 but also predict tertiary structure (e.g. pseudoknots). Most of the published uses of
 these tools has been focused on thier application to a research topic, rather than
 the implementation of a broad range methodology and workflow for designing 
 RNA-based circuits ~\cite{genetic_switchboard}. An article published last year
 by Keasling's lab states that, with their work, they sought to `establish a foundation for 
 developing computer-aided design platforms' ~\cite{Keasling_Model-Driven}.
 With that in mind, we seek to implement a CAD platform for the compilation of
 biocircuits, that synthesizes the many tools, and techniques that have already
 been detailed, concerning the modeling of riboregulation. The tool itself will
 be similar to simple electrical circuit design tools, and will generate
 algorithmically optimized compositors that can be used to construct a given circuit.
 As an exmaple of the tool's viability, one of the circuits discussed in class, that
 was previously implemented with a repressor-protein/promoter methodology, will be 
 be modeled with a riboregulatory methodology.

\newpage
  
\begin{thebibliography}{9}

\bibitem{review}Chappell, J., Takahashi, M. K., Meyer, S., Loughrey, D., Watters
  , K. E. and Lucks, J.(2013), 
  The centrality of RNA for engineering gene expression. 
  Biotechnology Journal. doi: 10.1002/biot.201300018

\bibitem{howfolds}Ignacio Tinoco Jr, Carlos Bustamante, How RNA folds, 
  Journal of Molecular Biology, Volume 293, Issue 2, 22 October 1999, 
  Pages 271-281, ISSN 0022-2836, http://dx.doi.org/10.1006/jmbi.1999.3001.

\bibitem{automated_design}Guillermo Rodrigo, Thomas E. Landrain, and Alfonso Jaramillo
  De novo automated design of small RNA circuits for engineering synthetic riboregulation in living cells
  PNAS 2012 109 (38) 15271-15276, doi:10.1073/pnas.1203831109

\bibitem{genetic_switchboard}Jarred M. Callura, Charles R. Cantor, and James J. Collins
  Genetic switchboard for synthetic biology applicationsz
  PNAS 2012 109 (15) 5850-5855; published ahead of print March 27, 2012, doi:10.1073/pnas.1203808109

\bibitem{Keasling_Model-Driven}Model-Driven Engineering of RNA Devices to Quantitatively Program Gene Expression
    James M. Carothers, Jonathan A. Goler, Darmawi Juminaga, and Jay D. Keasling
    Science 23 December 2011: 334 (6063), 1716-1719. [DOI:10.1126/science.1212209]


\end{thebibliography}

\end{document}
